% Clase y configuracion de tipo de documento
\documentclass[10pt,a4paper,spanish]{article}
% Inclusion de paquetes
\usepackage{a4wide}
\usepackage{amsmath, amscd, amssymb, amsthm, latexsym}
\usepackage[spanish]{babel}
\usepackage[utf8]{inputenc}
\usepackage[width=15.5cm, left=3cm, top=2.5cm, height= 24.5cm]{geometry}
\usepackage{fancyhdr}
\pagestyle{fancyplain}
\usepackage{listings}
\usepackage{enumerate}
\usepackage{xspace}
\usepackage{longtable}
\usepackage{caratula}
% incluye macros espec materia
%practicas
\newcommand{\practica}[2]{%
    \title{Pr\'actica #1 \\ #2}
    \author{Algoritmos y Estructura de Datos I}
    \date{Primer Cuatrimestre 2013}

    \maketitle
}

%ejercicios
\newtheorem{exercise}{Ejercicio}
\newenvironment{ejercicio}{\begin{exercise}\rm}{\end{exercise} \vspace{0.2cm}}
\newenvironment{items}{\begin{enumerate}[i)]}{\end{enumerate}}
\newenvironment{subitems}{\begin{enumerate}[a)]}{\end{enumerate}}
\newcommand{\sugerencia}[1]{\noindent \textbf{Sugerencia:} #1}

%tipos basicos
\newcommand{\rea}{\mathsf{Float}}
\newcommand{\float}{\mathbb{R}}
\newcommand{\ent}{\mathbb{Z}}
\newcommand{\cha}{\mathsf{Char}}
\newcommand{\bool}{\mathsf{Bool}}
\newcommand{\Then}{\rightarrow}
\newcommand{\Iff}{\leftrightarrow}

\newcommand{\mcd}{\mathrm{mcd}}
\newcommand{\prm}[1]{\ensuremath{\mathsf{prm}(#1)}}
\newcommand{\sgd}[1]{\ensuremath{\mathsf{sgd}(#1)}}
\newcommand{\trc}[1]{\ensuremath{\mathsf{trc}(#1)}}

%listas
\newcommand{\TLista}[1]{[#1]}
\newcommand{\lvacia}{\ensuremath{[\ ]}}
\newcommand{\longitud}[1]{\left| #1 \right|}
\newcommand{\cons}[1]{\mathsf{cons}(#1)}
%\newcommand{\cons}[1]{\mathrm{cons}(#1)}
\newcommand{\indice}[1]{\mathsf{indice}(#1)}
%\newcommand{\indice}[1]{\mathrm{indice}(#1)}
\newcommand{\conc}[1]{\mathsf{conc}(#1)}
%\newcommand{\conc}[1]{\mathrm{conc}(#1)}
\newcommand{\cab}[1]{\mathsf{cab}(#1)}
%\newcommand{\cab}[1]{\mathrm{cab}(#1)}
\newcommand{\cola}[1]{\mathsf{cola}(#1)}
%\newcommand{\cola}[1]{\mathrm{cola}(#1)}
\newcommand{\sub}[1]{\mathsf{sub}(#1)}
%\newcommand{\sub}[1]{\mathrm{sub}(#1)}
\newcommand{\en}[1]{\mathsf{en}(#1)}
%\newcommand{\en}[1]{\mathrm{en}(#1)}
\newcommand{\cuenta}[2]{\mathsf{cuenta}\ensuremath{(#1, #2)}}
%\newcommand{\cuenta}[2]{\ensuremath{\mathrm{cuenta}(#1, #2)}}
\newcommand{\suma}[1]{\mathsf{suma}(#1)}
%\newcommand{\suma}[1]{\mathrm{suma}(#1)}
\newcommand{\twodots}{\mathrm{..}}

% Acumulador
\newcommand{\acum}[1]{\mathsf{acum}(#1)}

% \selector{variable}{dominio}
\newcommand{\selector}[2]{#1~\ensuremath{\leftarrow}~#2}
\newcommand{\selec}{\ensuremath{\leftarrow}}

% -----------------
% Especificacion
% -----------------

% Para problemas con resultado:
% \begin{problema}{nombre}{argumentos}{resultado}
%       \modifica{variables}
%       \requiere[nombre]{condicion}
%       \requiere[nombre]{condicion}
%       ...
%       \asegura[nombre]{condicion}
%       \asegura[nombre]{condicion}
%       ...
% \end{problema}

% Para problemas sin resultado:
% \begin{problema*}{nombre}{argumentos}
%       \modifica{variables}
%       \requiere[nombre]{condicion}
%       \requiere[nombre]{condicion}
%       ...
%       \asegura[nombre]{condicion}
%       \asegura[nombre]{condicion}
%       ...
% \end{problema*}

\newenvironment{problema}[3]{
    \vspace{0.2cm}
    \noindent \textsf{problema #1}\ensuremath{(#2) = #3\{}\\
    \begin{tabular}{p{0.02\textwidth} p{0.85 \textwidth}}
}{
    \end{tabular}

    \noindent \ensuremath{\}}
    \vspace{0.15cm}
}

\newenvironment{problema*}[2]{
    \vspace{0.2cm}
    \noindent \textsf{problema #1}\ensuremath{(#2)\{}\\
    \begin{tabular}{p{0.02\textwidth} p{0.85 \textwidth}}
}{
    \end{tabular}

    \noindent \ensuremath{\}}
    \vspace{0.15cm}
}

\newcommand{\requiere}[2][]{& \textsf{requiere #1: }\ensuremath{#2};\\}
\newcommand{\asegura}[2][]{& \textsf{asegura #1: }\ensuremath{#2};\\}
\newcommand{\modifica}[1]{& \textsf{modifica }\ensuremath{#1};\\}
\newcommand{\pre}[1]{\textsf{pre}\ensuremath{(#1)}}
\newcommand{\aux}[2]{& \textsf{aux }\ensuremath{#1 = #2};\\}
\newcommand{\problemanom}[1]{\textsf{#1}}
\newcommand{\problemail}[3]{\textsf{problema #1}\ensuremath{(#2) = #3}}
\newcommand{\problemailsinres}[2]{\textsf{problema #1}\ensuremath{(#2)}}
\newcommand{\requiereil}[2][]{\textsf{requiere #1: }\ensuremath{#2}}
\newcommand{\asegurail}[2][]{\textsf{asegura #1: }\ensuremath{#2}}
\newcommand{\modificail}[1]{\textsf{modifica }\ensuremath{#1}}
\newcommand{\auxil}[2]{\textsf{aux }\ensuremath{#1 = #2};}
\newcommand{\auxnom}[1]{\textsf{aux }\ensuremath{#1}}

% -----------------
% Tipos compuestos
% -----------------

\newcommand{\Pred}[1]{\mathit{#1}}
\newcommand{\TSet}[1]{\textsf{Conjunto}\ensuremath{\langle #1 \rangle}}
\newcommand{\TSetFinito}[1]{\textsf{Conjunto}\ensuremath{\langle #1 \rangle}}
\newcommand{\TRac}{\tiponom{Racional}}
\newcommand{\TVec}{\tiponom{Vector}}
\newcommand{\True}{\mathrm{True}}
\newcommand{\False}{\mathrm{False}}
\newcommand{\Func}[1]{\mathrm{#1}}
\newcommand{\cardinal}[1]{\left| #1 \right|}
\newcommand{\enum}[2]{\ensuremath{\mathsf{#1} = \langle \mathsf{#2} \rangle}}

\newenvironment{tipo}[1]{%
    \vspace{0.2cm}
    \textsf{tipo #1}\ensuremath{\{}\\
    \begin{tabular}[l]{p{0.02\textwidth} p{0.02\textwidth} p{0.82 \textwidth}}
}{%
    \end{tabular}

    \ensuremath{\}}
    \vspace{0.15cm}
}

\newcommand{\observador}[3]{%
    & \multicolumn{2}{p{0.85\textwidth}}{\textsf{observador #1}\ensuremath{(#2):#3}}\\%
}
\newcommand{\observadorconreq}[3]{
    & \multicolumn{2}{p{0.85\textwidth}}{\textsf{observador #1}\ensuremath{(#2):#3 \{}}\\
}
\newcommand{\observadorconreqfin}{
    & \multicolumn{2}{p{0.85\textwidth}}{\ensuremath{\}}}\\
}
\newcommand{\obsrequiere}[2][]{& & \textsf{requiere #1: }\ensuremath{#2};\\}

\newcommand{\explicacion}[1]{&& #1 \\}
\newcommand{\invariante}[1]{%
    & \multicolumn{2}{p{0.85\textwidth}}{\textsf{invariante }\ensuremath{#1}}\\%
}
\newcommand{\auxinvariante}[2]{
    & \multicolumn{2}{p{0.85\textwidth}}{\textsf{aux }\ensuremath{#1 = #2}};\\
}

\newcommand{\tiponom}[1]{\ensuremath{\mathsf{#1}}\xspace}
\newcommand{\obsnom}[1]{\ensuremath{\mathsf{#1}}}

% -----------------
% Ecuaciones de terminacion en funcional
% -----------------

\newenvironment{ecuaciones}{%
    $$
    \begin{array}{l @{\ /\ (} l @{,\ } l @{)\ =\ } l}
}{%
    \end{array}
    $$
}

\newcommand{\ecuacion}[4]{#1 & #2 & #3 & #4\\}

% Listas por comprension. El primer parametro es la expresion y el
% segundo tiene los selectores y las condiciones.
\newcommand{\comp}[2]{[\,#1\,|\,#2\,]}


% Encabezado
\lhead{Algoritmos y Estructuras de Datos I}
\rhead{Grupo 4}
% Pie de pagina
\renewcommand{\footrulewidth}{0.4pt}
\lfoot{Facultad de Ciencias Exactas y Naturales}
\rfoot{Universidad de Buenos Aires}

\begin{document}

% Datos de caratula
\materia{Algoritmos y Estructuras de Datos I}
\titulo{Trabajo Pr\'actico N\'umero 1}
\subtitulo{Especificaci\'on}
\grupo{Grupo: 4}

\integrante{Aun Castells, Mar\'ia Virginia}{366/13}{vauncastells@hotmail.com}
\integrante{Motta, Leandro}{85/14}{leamotta@msn.com}
\integrante{Zdanovitch, Nikita}{520/14}{3hb.tch@gmail.com}
\integrante{de Monasterio, Francisco}{764/13}{franciscodemonasterio@outlook.com}

\maketitle

\newpage

% Para crear un indice
%\tableofcontents

% Forzar salto de pagina
\clearpage

% Pueden modularizar el documento incluyendo otros .tex
% \include{observaciones}

\section{Observaciones}

	\begin{enumerate}
		\item un item
		\item otro item
	\end{enumerate}

% Otro salto de pagina
% \newpage

\section{Resolución}

\begin{ejercicio}
	% EJEMPLO
	Blur:

	\begin{problema}{blur}{imagen:[[(\ent,\ent,\ent)]], k:\ent}{res:[[(\ent,\ent,\ent)]]}
		\requiere[kEsPositivo]{k > 0}
		\requiere[entradaEsRectangular]{EsRectangular(imagen)}
		\requiere[todosLosPixelsSonValidos]{(\forall f \selec imagen) (\forall px \selec f) ()}
		\asegura[dimensionesDeSalidaIguales]{Alto(res) == Alto(imagen) \wedge Ancho(res) == Ancho(imagen)}
		\asegura[aseguraTodo]{(\forall y \selec [0:\mid Alto(imagen)\mid)) (\forall x \selec [0:\mid Ancho(imagen)\mid))\ if\  esKCompleto(KVecinos(imagen, y, x, k), k)\ then\ esPromedio(res, imagen, y, x, k)\ else\ esNegativo(res, y, x)}
	\end{problema}

\end{ejercicio}

\begin{ejercicio}
	% EJEMPLO
	Acuarela:

	\begin{problema}{acuarela}{imagen:[[(\ent,\ent,\ent)]], k:\ent}{res:[[(\ent,\ent,\ent)]]}
		\requiere[kEsPositivo]{k > 0}
		\requiere[entradaEsRectangular]{EsRectangular(imagen)}
		\asegura[dimensionesDeSalidaIguales]{Alto(res) == Alto(imagen)\ \wedge\ Ancho(res) == Ancho(imagen)}
		\asegura[efecto]{(\forall i \selec [0:\longitud{Alto(res)})) (\forall j \selec [0:\longitud{Ancho(res)}))\ res[i][j] == MedianaONegro(i,j,img,k)}
	\end{problema}
	
\end{ejercicio}

\begin{ejercicio}
	% EJEMPLO
	Dividir:

	\begin{problema}{dividir}{imagen:[[(\ent,\ent,\ent)]], m,n:\ent}{res:[[(\ent,\ent,\ent)]]}
		\requiere[nYmEsPositivo]{n > 0 \wedge m > 0}
		\requiere[entradaTieneSuperficie]{Alto(imagen) > 0\ \wedge\  Ancho(imagen > 0}
		\requiere[todosLosPixelsSonValidos]{(\forall f \selec imagen) (\forall px \selec f)}
		\requiere[divideEnFilasIguales]{Alto(imagen)\mod{m} == 0}
		\requiere[divideEnColumnasIguales]{Ancho(imagen)\mod{n} == 0}		
		\asegura{mismo(res,SepararHorizontal(SepararVertical(imagen,n),m))}
	\end{problema}

\end{ejercicio}

\begin{ejercicio}
	% EJEMPLO
	Pegar:

	\begin{problema}{pegar}{imagen,imagen2:[[(\ent,\ent,\ent)]],\ pixel:(\ent ,\ent ,\ent}{}
		\modifica{imagen}
		\requiere[hayRectanguloColorPixelEnImagen]{hayRectangulo(pre(imagen),pixel)}
		\requiere[entradaEsRectangular]{EsRectangular(pre(imagen))}
		\requiere[entrada2EsRectangular]{EsRectangular(imagen2)}
		\requiere[pixelesFormanRectangulo]{EsRectangular(rectaPixel(pre(imagen),pixel))}
		\requiere[entradaTieneSuperficie]{Alto(pre(imagen)) > 0\ \wedge\  Ancho(pre(imagen) > 0}
		\requiere[entrada2TieneSuperficie]{Alto(imagen2) > 0\ \wedge\  Ancho(imagen2 > 0}
		\asegura[imagen2ContenidaEnRect]{if\ Ancho(imagen2) \leq Ancho(rectaPixel(pre(imagen), pixel))\ \wedge\ Alto(imagen2) \leq Alto(rectaPixel(pre(imagen), pixel))\ then\ imagen == pegoImagenes(pre(imagen),\ pegoImagenes(rectaPixel(pre(imagen), pixel),\ imagen2,\ 0,\ 0),\ ,\ )\ else\ pre(imagen) == imagen}
		\asegura[imagenTieneMismoTamaño]{Alto(pre(imagen)) == Alto(imagen) \wedge Ancho(pre(imagen)) == Ancho(imagen)}
	\end{problema}

\end{ejercicio}

%\begin{ejercicio}
%	% EJEMPLO
%	Otro ejemplo sin resultado:
	
%	\begin{problema*}{cociente}{a,b:\ent}
%		\requiere{b \neq 0}
%		\modifica{a,b}
%		\asegura{a == pre(a)$ $div$ $pre(b)}
%		\asegura{b == pre(a)$ $mod$ $pre(b)}
%	\end{problema*}

%\end{ejercicio}

% \subsection{Ejercicio X}

\subsection{Auxiliares}

%\begin{itemize}
%	\item \auxil{Alto(img:{{(\ent ,\ent ,\ent)}}} : \ent}\longitud{img}}
%\end{itemize}

\begin{itemize}
	\item \auxil{Alto(img:[[(\ent ,\ent ,\ent)]] : \ent}{ \longitud{img}}
\end{itemize}

\begin{itemize}
	\item \auxil{Ancho(img:[[(\ent ,\ent ,\ent)]]) : \ent}{if\ \longitud{img} == 0\  then\ 0\ else \longitud{cab(img)}}
\end{itemize}

\begin{itemize}
	\item \auxil{esIndiceValido(y,x: \ent , img:[[(\ent ,\ent ,\ent)]] : \bool}{0 \leqslant y < Alto(img) \wedge 0 \leqslant x < Ancho(img)}
\end{itemize}

\begin{itemize}
	\item \auxil{esRectangular(img:[[(\ent ,\ent ,\ent)]] : \bool}{(\forall a\selec img) \longitud{a} == \longitud{cab(img)}}
\end{itemize}

\begin{itemize}
	\item \auxil{KIndices(y,x,k: \ent : [(\ent ,\ent)] }{[(i,j)\ |\ i\selec(y-k:y+k), i\selec(y-k:y+k)]}
\end{itemize}

\begin{itemize}
	\item \auxil{KVecinos(img:[[(\ent ,\ent ,\ent)]], y,x,k: \ent : [(\ent ,\ent ,\ent)]}{[Pixel(img, \prm{c}, \sgd{c})\ |\ c\selec KIndices(y,x,k),\  esIndiceValido(c)]}
\end{itemize}

\begin{itemize}
	\item \auxil{Pixel(img:[[(\ent ,\ent ,\ent)]], y,x:\ent) : (\ent ,\ent ,\ent)}{if\ esIndiceValido(y, x, img)\ then\ img[y][x]\ else (0, 0, 0)}
\end{itemize}

\begin{itemize}
	\item \auxil{EsNegro(r:[[(\ent ,\ent ,\ent)]], y,x:\ent) : \bool}{\prm{Pixel(r,y,x} == 0 \wedge \sgd{Pixel(r,y,x} == 0 \wedge \trc{Pixel(r,y,x} == 0}
\end{itemize}

\begin{itemize}
	\item \auxil{EsPromedio(r,img:[[(\ent ,\ent ,\ent)]], y,x,k:\ent) : \bool}{\prm{Pixel(r,y,x)} == (\sum{\prm{p}\ |\ p\selec KVecinos(img,y,x,k) \longitud{KVecinos(img,y,x,k)}} \wedge \sgd{Pixel(r,y,x)} == (\sum{\sgd{p}\ |\ p\selec KVecinos(img,y,x,k) \longitud{KVecinos(img,y,x,k)}} \wedge \trc{Pixel(r,y,x)} == (\sum{\trc{p}\ |\ p\selec KVecinos(img,y,x,k) \longitud{KVecinos(img,y,x,k)}}}
\end{itemize}

\begin{itemize}
	\item \auxil{EsKCompleto(kv:[(\ent ,\ent ,\ent)],\ k:\ent) : \bool}{\longitud{kv} == (k+k-1)^2}
\end{itemize}

\begin{itemize}
	\item \auxil{VektorKVecinos(img:[[(\ent ,\ent ,\ent)]],\ i,j,k: \ent) : [(\ent ,\ent ,\ent)]}{[img[a][b]\ |\ a\selec [i-k+1:i+k-1],\ b\selec[j-k+1:j+k-1],\ a\geq 0\ \wedge\ b\geq 0\ \wedge\ a < Alto(img)\ \wedge\ b < Ancho(img)]}
\end{itemize}

\begin{itemize}
	\item \auxil{MedianaONegro(img:[[(\ent ,\ent ,\ent)]],\ i,j,k: \ent) : (\ent ,\ent ,\ent)}{if\ esKCompleto(vectorKvecinos(img,k,i,j))\ then\ mediana(vectorKvecinos(img,k,i,j))\ else (0,0,0)}
\end{itemize}

\begin{itemize}
	\item \auxil{mediana(vector:[(\ent ,\ent ,\ent)]) : (\ent ,\ent ,\ent)}{(valorMediana([\prm{a}\ |\ a\selec vector],\ valorMediana([\sgd{a}\ |\ a\selec vector],\ valorMediana([\trc{a}\ |\ a\selec vector])}
\end{itemize}

\begin{itemize}
	\item \auxil{valorMediana(xs:[\ent]) : \ent}{enOrden(xs)[\longitud{xs}\ $div$\ 2]}
\end{itemize}

\begin{itemize}
	\item \auxil{enOrden(xs:[\ent]) : [\ent]}{[x\ |\ i\selec [0:\longitud{xs}),\ x\selec xs,\ cuentaMenores(xs,x) == i]}
\end{itemize}

\begin{itemize}
	\item \auxil{cuentaMenores(xs:[\ent],\ x:\ent) : \ent}{\longitud{[1\ |\ y\ xs,\ y < x]}}
\end{itemize}

\begin{itemize}
	\item \auxil{SepararVertical(img:[[(\ent ,\ent ,\ent)]],\ columnas: \ent) : [[[(\ent ,\ent ,\ent)]]]}{[verticalizarImagen(img,columnas)[i*Alto(img)..(i+1)*Alto(img))\ |\ i\selec[0..columnas)]}
\end{itemize}

\begin{itemize}
	\item \auxil{verticalizarImagen(img:[[(\ent ,\ent ,\ent)]],\ columnas: \ent) : [[(\ent ,\ent ,\ent)]]}{[img[i][Ancho(img)*k\ $div$\ columnas..Ancho(img)*(k+1\ $div$\ columnas)]\ |\ k\selec [0..columnas),\ i\selec[0..Alto(img))]}
\end{itemize}

\begin{itemize}
	\item \auxil{SepararHorizontal(listaimg:[[[(\ent ,\ent ,\ent)]]],\ filas: \ent) : [[[(\ent ,\ent ,\ent)]]]}{[listaimg[i][\longitud{cab(listaimg)}*k\ $div$\ filas..\longitud{cab(listaimg)}*(k+1)\ $div$\ filas\ |\ k\selec[0..filas),\ i\selec[0..\longitud{listaimg})]\ |\ a\selec [i-k+1:i+k-1],\ b\selec[j-k+1:j+k-1],\ a\geq 0\ \wedge\ b\geq 0\ \wedge\ a < Alto(img)\ \wedge\ b < Ancho(img)]}
\end{itemize}

\begin{itemize}
	\item \auxil{cuenta(x:T,\ a:\TLista{T}) : \ent}{long([y\ |\ y\selec a,\ y == x])}
\end{itemize}

\begin{itemize}
	\item \auxil{mismos(a,b:\TLista{T}) : \bool}{(\longitud{a}==\longitud{b}\ \wedge\ (\forall c\selec a)\ cuenta(c,a) == cuenta(c,b))}
\end{itemize}

\begin{itemize}
	\item \auxil{hayRectangulo(img:[[(\ent ,\ent ,\ent)]],\ pxl: (\ent ,\ent ,\ent)) : \bool}{i\selec[0..Alto(img)),\ j\selec[0..Ancho(img)),\ 4DelMismoColor(vectorKVecinos(img,2,i,j),\ pxl)}
\end{itemize}

\begin{itemize}
	\item \auxil{4DelMismoColor(kv:[(\ent ,\ent ,\ent)],\ pxl: (\ent ,\ent ,\ent)) : \bool}{(\forall f\selec kv)}
\end{itemize}
        
%\begin{itemize}
%	\item \auxil{inversa(a:\TLista{T}) : \TLista{T}}{[a_{|a|-i-1}\ |\ i \selec [0 \twodots \longitud{a})]}
%\end{itemize}
        
\end{document}
