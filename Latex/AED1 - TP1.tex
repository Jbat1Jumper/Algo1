% Clase y configuracion de tipo de documento
\documentclass[10pt,a4paper,spanish]{article}
% Inclusion de paquetes
\usepackage{a4wide}
\usepackage{mathtools, amscd, amssymb, amsthm, latexsym}
\usepackage[spanish]{babel}
\usepackage[utf8]{inputenc}
\usepackage[width=15.5cm, left=3cm, top=2.5cm, height= 24.5cm]{geometry}
\usepackage{fancyhdr}
\pagestyle{fancyplain}
\usepackage{listings}
\usepackage{enumerate}
\usepackage{xspace}
\usepackage{longtable}
\usepackage{caratula}
% incluye macros espec materia
\include{algo1-cmds}

% Encabezado
\lhead{Algoritmos y Estructuras de Datos I}
\rhead{Grupo 4}
% Pie de pagina
\renewcommand{\footrulewidth}{0.4pt}
\lfoot{Facultad de Ciencias Exactas y Naturales}
\rfoot{Universidad de Buenos Aires}

\begin{document}

% Datos de caratula
\materia{Algoritmos y Estructuras de Datos I}
\titulo{Trabajo Pr\'actico N\'umero 1}
\subtitulo{Especificaci\'on}
\grupo{Grupo: 4}

\integrante{Aun Castells, Mar\'ia Virginia}{366/13}{vauncastells@hotmail.com}
\integrante{Motta, Leandro}{85/14}{leamotta@msn.com}
\integrante{Zdanovitch, Nikita}{520/14}{3hb.tch@gmail.com}
\integrante{de Monasterio, Francisco}{764/13}{franciscodemonasterio@outlook.com}

\maketitle

\newpage

% Para crear un indice
%\tableofcontents

% Forzar salto de pagina
\clearpage

% Pueden modularizar el documento incluyendo otros .tex
% \include{observaciones}

%\section{Observaciones}
%
%	\begin{enumerate}
%		\item un item
%		\item otro item
%	\end{enumerate}
%
% Otro salto de pagina
% \newpage

\section{Resolución}

\begin{ejercicio}
	% EJEMPLO
	Blur:

	\begin{problema}{blur}{imagen:[[(\ent,\ent,\ent)]],\ k:\ent}{res:[[(\ent,\ent,\ent)]]}
		\requiere[kEsPositivo]{
			k > 0
		}
		\requiere{
			esRectangular(imagen)
		}
		\requiere[todosLosPixelsSonValidos]{
			(\forall\ f \selec imagen) (\forall\ px \selec f)\ esPixelValido(px)
		}
		\asegura[conservaDimenciones]{
			alto(res) == alto(imagen) \land ancho(res) == ancho(imagen)
		}
		\aseguraml{
			\left( \forall\ y \selec {[0\,..\,alto(imagen))} \right) \left( \forall\ x \selec {[0\,..\,ancho(imagen))} \right) 
		}
		\nuevalinea{ \IF   esKCompleto(kVecinos(imagen,\ y,\ x,\ k),\ k)\ }
		\nuevalinea{ \THEN esPromedio(res, imagen,\ y,\ x,\ k)\ }
		\nuevalinea{ \ELSE esNegro(res,\ y,\ x) ;}
	\end{problema}

\end{ejercicio}

\begin{ejercicio}
	% EJEMPLO
	Dividir:

	\begin{problema}{dividir}{imagen:[[(\ent,\ent,\ent)]],\ m,n:\ent}{res:[[[(\ent,\ent,\ent)]]]}
		\requiere[nYmEsPositivo]{
			n > 0 \wedge m > 0
		}
		\requiere{
			esRectangular(imagen)
		}
		\requiere{
			imagenTieneSuperficie(imagen)
		}
		\requiere[todosLosPixelsSonValidos]{
			(\forall\ f \selec imagen) (\forall\ px \selec f)\ esPixelValido(px)
		}
		\requiere[divideEnFilasIguales]{
			alto(imagen)\mod{m} == 0
		}
		\requiere[divideEnColumnasIguales]{
			ancho(imagen)\mod{n} == 0
		}		
		\asegura{
			mismo\left(res,\ separarHorizontal\left(separarVertical(imagen,\ n),\ m\right)\right)
		}
	\end{problema}

\end{ejercicio}

\begin{ejercicio}
	Acuarela:

	\begin{problema}{acuarela}{imagen:[[(\ent,\ent,\ent)]],\ k:\ent}{res:[[(\ent,\ent,\ent)]]}		
		\requiere[kEsPositivo]{
			k > 0
		}
		\requiere{
			esRectangular(imagen)
		}
		\asegura[conservaDimenciones]{
			alto(res) == alto(imagen) \land ancho(res) == ancho(imagen)
		}
		\aseguraml[]{
			( \forall\ i \selec [0\,..\,alto(res)),\ j \selec [0\,..\,ancho(res))) 
		}
		\nuevalinea{ res[i][j] == medianaONegro(i,\ j,\ img,\ k); }
	\end{problema}

\end{ejercicio}

\begin{ejercicio}

	Pegar:
	
	\begin{problema}{pegar}{...}{...}
		...
	\end{problema}

\end{ejercicio}

% \subsection{Ejercicio X}

\subsection{Auxiliares}

\begin{itemize}
	\item \auxil{esRectangular(img:[[(\ent,\ent,\ent)]]) : \bool}{
		(\forall a \selec img)\ |a| == |\cab{img}|
	}
	\item \auxil{esPixelValido(px:(\ent,\ent,\ent)) : \bool}{
		esByte(\prm{px}) \land esByte(\sgd{px}) \land esByte(\trc{px}) 
	}
	\item \auxil{esByte(b:\ent) : \bool}{
		0 \leq b \leq 255
	}	
	\item \auxil{alto(img:[[(\ent,\ent,\ent)]]) : \ent}{
		|img|
	}
	\item \auxil{ancho(img:[[(\ent,\ent,\ent)]]) : \ent}{
		\IF |img| == 0 \THEN 0 \ELSE |\cab{img}|
	}
	
	\item \auxil{esKCompleto(kv:[(\ent,\ent,\ent)],\ k:\ent) : \bool}{
		|kv| == \left(k+k-1\right)^2
	}
	
	\item \auxilml{esPromedio(res, img:[[(\ent,\ent,\ent)]],\ y,x,k:\ent) : \bool}{}
	\auxnuevalinea{ 
		\prm{pixel(res,\ y,\ x)} == 
		\left({\mathsf{sum}([\prm{p}\ |\ p\selec kVecinos(img,\ y,\ x,\ k)])\ /\ |kVecinos(img,\ y,\ x,\ k)|}\right) \land 
	}
	\auxnuevalinea{ 
		\ \sgd{pixel(res,\ y,\ x)} == 
		\left({\mathsf{sum}([\sgd{p}\ |\ p\selec kVecinos(img,\ y,\ x,\ k)])\ /\ |kVecinos(img,\ y,\ x,\ k)|}\right) \land 
	}
	\auxnuevalinea{ 
		\ \sgd{pixel(res,\ y,\ x)} == 
		\left({\mathsf{sum}([\trc{p}\ |\ p\selec kVecinos(img,\ y,\ x,\ k)])\ /\ |kVecinos(img,\ y,\ x,\ k)|}\right);
	}
		
	\item \auxil{esNegro(img:[[(\ent,\ent,\ent)]],\ y,x:\ent) : \bool}{
		prm(pixel(img,\ y,\ x)) == 0\ \land\ \sgd{pixel(img,\ y,\ x)} == 0\ \land\ \trc{pixel(img,\ y,\ x)} == 0 
	}
	
	\item \auxil{pixel(img:[[(\ent,\ent,\ent)]],\ y,x:\ent) : (\ent,\ent,\ent)}{
		 \IF esIndiceValido(y,\ x,\ img) \THEN img[y][x] \ELSE (0,\ 0,\ 0)
	}
	\item \auxil{kVecinos(img:[[(\ent,\ent,\ent)]],\ y,x,k:\ent) : [(\ent,\ent,\ent)]}{
		[pixel(img,\ prm(c),\ sgd(c))\ |\ c \selec kIndices(y,\ x,\ k),\ esIndiceValido(c)]
	}
	\item \auxil{kIndices(y,x,k:\ent) : [(\ent,\ent)]}{
		[(i,\ j)\ |\ i \selec (y-k\,..\,y+k),\ j \selec (x-k\,..\,x+k)]
	}
	
	\item \auxil{esIndiceValido(y,x:\ent,\ img:[[(\ent,\ent,\ent)]]) : \bool}{
		0 \leq y < alto(img) \land 0 \leq x < ancho(img)
	}
	
	
	%acuarela


	\item \auxilml{medianaONegro(i,j:\ent,\ img:[[(\ent,\ent,\ent)]],\ k:\ent) : (\ent,\ent,\ent)}{	}
	\auxnuevalinea{ 
		\IF esKCompleto(kVecinos(img,\ i,\ j,\ k),\ k) 
	}
	\auxnuevalinea{ 
		\THEN mediana(kVecinos(img,\ i,\ j,\ k)) 
	}
	\auxnuevalinea{ 
		\ELSE (0,\ 0,\ 0);
	}
	
	\item \auxilml{mediana(kv:[(\ent,\ent,\ent)]) : (\ent,\ent,\ent)}{}
	\auxnuevalinea{ 
		( valorMediana([ \prm{a}\ |\ a \selec kv]),
	}
	\auxnuevalinea{ 
		valorMediana([ \sgd{a}\ |\ a \selec kv]),	
	}
	\auxnuevalinea{ 
		valorMediana([ \trc{a}\ |\ a \selec kv]));
	}
	
	\item \auxil{valorMediana(xs:[\ent]) : \ent}{
		enOrden(xs)[|xs|/2]
	}
	
	\item \auxil{enOrden(xs:[\ent]) : [\ent]}{
		[ x\ |\ i \selec [0\,..\,|xs|),\ x \selec xs,\ cuentaMenores(xs,\ x) == i ]
	}
	
	\item \auxil{cuentaMenores(xs:[\ent],\ x:\ent) : \ent}{
		|[1\ |\ y \selec xs,\ y < x ]|
	}
	
	% dividir
	
	\item \auxilml{separarVertical(img:[[(\ent,\ent,\ent)]],\ columnas:\ent) : [[[(\ent,\ent,\ent)]]]}{}
	\auxnuevalinea{
		\big[verticalizarImagen(img,\ columnas)[alto(img) i\,..\,alto(img)(i+1))\ \big|\ i \selec [0\,..\,columnas) \big];
	}
	
	\item \auxilml{verticalizarImagen(img:[[(\ent,\ent,\ent)]],\ columnas:\ent) : [[(\ent,\ent,\ent)]]}{}
	\auxnuevalinea{
		\big[img[i][Ancho(img)k/columnas\,..\,Ancho(img)(k+1)/columnas)\ \big|\ k \selec [0\,..\,columnas),\ i \selec [0\,..\,Alto(img))\big];
	}
		
	\item \auxilml{verticalizarImagen(listaimg:[[[(\ent,\ent,\ent)]]],\ filas:\ent) : [[[(\ent,\ent,\ent)]]]}{}
	\auxnuevalinea{
		\big[  listaimg[i][|cab(listaimg)|k/filas\,..\,|cab(listaimg)|(k+1)/filas)\ \big|\ k \selec [0\,..\,filas),\ i \selec [0\,..\,|listaimg|)  \big];
	}
	
	\item \auxil{cuenta(x:T,\ a:[T]) : \ent}{
		\big|[ y\ |\ y \selec a,\ y == x ]\big|
	}
	
	\item \auxil{mismos(a,b:[T]) : \bool}{
		(|a| == |b|) \land ( \forall\ c \selec a)\ cuenta(c,\ a) == cuenta(c,\ b)
	}
	
%  aux SepararHorizontal (listaimg:[[[(Int,Int,Int)]]] , filas : Int) : [[[(Int,Int,Int)]]] =
%  [listaimg[i][|cab(listaimg)|*k/filas .. |cab(listaimg)|*(k+1)/filas) | k<-[0..filas),i<-[0..|listaimg|)];
	
	
	% muestra
%	\item \auxil{inversa(a:\TLista{T}) : \TLista{T}}{
%		[a_{|a|-i-1}\ |\ i \selec [0 \twodots \longitud{a})]
%	}
\end{itemize}
        
\end{document}
