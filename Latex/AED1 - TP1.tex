% Clase y configuracion de tipo de documento
\documentclass[10pt,a4paper,spanish]{article}
% Inclusion de paquetes
\usepackage{a4wide}
\usepackage{amsmath, amscd, amssymb, amsthm, latexsym}
\usepackage[spanish]{babel}
\usepackage[utf8]{inputenc}
\usepackage[width=15.5cm, left=3cm, top=2.5cm, height= 24.5cm]{geometry}
\usepackage{fancyhdr}
\pagestyle{fancyplain}
\usepackage{listings}
\usepackage{enumerate}
\usepackage{xspace}
\usepackage{longtable}
\usepackage{caratula}
% incluye macros espec materia
\include{algo1-cmds}

% Encabezado
\lhead{Algoritmos y Estructuras de Datos I}
\rhead{Grupo 4}
% Pie de pagina
\renewcommand{\footrulewidth}{0.4pt}
\lfoot{Facultad de Ciencias Exactas y Naturales}
\rfoot{Universidad de Buenos Aires}

\begin{document}

% Datos de caratula
\materia{Algoritmos y Estructuras de Datos I}
\titulo{Trabajo Pr\'actico N\'umero 1}
\subtitulo{Especificaci\'on}
\grupo{Grupo: 4}

\integrante{Aun Castells, Mar\'ia Virginia}{366/13}{vauncastells@hotmail.com}
\integrante{Motta, Leandro}{85/14}{leamotta@msn.com}
\integrante{Zdanovitch, Nikita}{520/14}{3hb.tch@gmail.com}
\integrante{de Monasterio, Francisco}{764/13}{franciscodemonasterio@outlook.com}

\maketitle

\newpage

% Para crear un indice
%\tableofcontents

% Forzar salto de pagina
\clearpage

% Pueden modularizar el documento incluyendo otros .tex
% \include{observaciones}

\section{Observaciones}

	\begin{enumerate}
		\item un item
		\item otro item
	\end{enumerate}

% Otro salto de pagina
% \newpage

\section{Resolución}

\begin{ejercicio}
	% EJEMPLO
	Blur:

	\begin{problema}{blur}{imagen:[[(\ent,\ent,\ent)]], k:\ent}{res:[[(\ent,\ent,\ent)]]}
		\requiere[kEsPositivo]{k > 0}
		\requiere[entradaEsRectangular]{EsRectangular(imagen)}
		\requiere[todosLosPixelsSonValidos]{(\forall f \selec imagen) (\forall px \selec f) ()}
		\asegura[dimensionesDeSalidaIguales]{Alto(res) == Alto(imagen) \wedge Ancho(res) == Ancho(imagen)}
		\asegura[aseguraTodo]{(\forall y \selec [0:\mid Alto(imagen)\mid)) (\forall x \selec [0:\mid Ancho(imagen)\mid))\ if\  esKCompleto(KVecinos(imagen, y, x, k), k)\ then\ esPromedio(res, imagen, y, x, k)\ else\ esNegativo(res, y, x)}
	\end{problema}

\end{ejercicio}

\begin{ejercicio}
	% EJEMPLO
	Dividir:

	\begin{problema}{dividir}{imagen:[[(\ent,\ent,\ent)]], m,n:\ent}{res:[[(\ent,\ent,\ent)]]}
		\requiere[nYmEsPositivo]{n > 0 \wedge m > 0}
		\requiere[entradaTieneSuperficie]{Alto(imagen) > 0\ \wedge\  Ancho(imagen > 0}
		\requiere[todosLosPixelsSonValidos]{(\forall f \selec imagen) (\forall px \selec f)}
		\requiere[divideEnFilasIguales]{Alto(imagen)\mod{m} == 0}
		\requiere[divideEnColumnasIguales]{Ancho(imagen)\mod{n} == 0}		
		\asegura{mismo(res,SepararHorizontal(SepararVertical(imagen,n),m))}
	\end{problema}

\end{ejercicio}

\begin{ejercicio}
	% EJEMPLO
	Blur:

	\begin{problema}{cociente}{a,b:\ent}{res:\ent}
		\requiere{b \neq 0}
		\asegura{res == a$ $div$ $b}
	\end{problema}

\end{ejercicio}

\begin{ejercicio}
	% EJEMPLO
	Otro ejemplo sin resultado:
	
	\begin{problema*}{cociente}{a,b:\ent}
		\requiere{b \neq 0}
		\modifica{a,b}
		\asegura{a == pre(a)$ $div$ $pre(b)}
		\asegura{b == pre(a)$ $mod$ $pre(b)}
	\end{problema*}

\end{ejercicio}

% \subsection{Ejercicio X}

\subsection{Auxiliares}

\begin{itemize}
	\item \auxil{inversa(a:\TLista{T}) : \TLista{T}}{[a_{|a|-i-1}\ |\ i \selec [0 \twodots \longitud{a})]}
\end{itemize}
        
\end{document}
