%practicas
\newcommand{\practica}[2]{%
    \title{Pr\'actica #1 \\ #2}
    \author{Algoritmos y Estructura de Datos I}
    \date{Primer Cuatrimestre 2013}

    \maketitle
}

%ejercicios
\newtheorem{exercise}{Ejercicio}
\newenvironment{ejercicio}{\begin{exercise}\rm}{\end{exercise} \vspace{0.2cm}}
\newenvironment{items}{\begin{enumerate}[i)]}{\end{enumerate}}
\newenvironment{subitems}{\begin{enumerate}[a)]}{\end{enumerate}}
\newcommand{\sugerencia}[1]{\noindent \textbf{Sugerencia:} #1}

%tipos basicos
\newcommand{\rea}{\mathsf{Float}}
\newcommand{\float}{\mathbb{R}}
\newcommand{\ent}{\mathbb{Z}}
\newcommand{\cha}{\mathsf{Char}}
\newcommand{\bool}{\mathsf{Bool}}
\newcommand{\Then}{\rightarrow}
\newcommand{\Iff}{\leftrightarrow}

\newcommand{\mcd}{\mathrm{mcd}}
\newcommand{\prm}[1]{\ensuremath{\mathsf{prm}(#1)}}
\newcommand{\sgd}[1]{\ensuremath{\mathsf{sgd}(#1)}}
\newcommand{\trc}[1]{\ensuremath{\mathsf{trc}(#1)}}

%listas
\newcommand{\TLista}[1]{[#1]}
\newcommand{\lvacia}{\ensuremath{[\ ]}}
\newcommand{\longitud}[1]{\left| #1 \right|}
\newcommand{\cons}[1]{\mathsf{cons}(#1)}
%\newcommand{\cons}[1]{\mathrm{cons}(#1)}
\newcommand{\indice}[1]{\mathsf{indice}(#1)}
%\newcommand{\indice}[1]{\mathrm{indice}(#1)}
\newcommand{\conc}[1]{\mathsf{conc}(#1)}
%\newcommand{\conc}[1]{\mathrm{conc}(#1)}
\newcommand{\cab}[1]{\mathsf{cab}(#1)}
%\newcommand{\cab}[1]{\mathrm{cab}(#1)}
\newcommand{\cola}[1]{\mathsf{cola}(#1)}
%\newcommand{\cola}[1]{\mathrm{cola}(#1)}
\newcommand{\sub}[1]{\mathsf{sub}(#1)}
%\newcommand{\sub}[1]{\mathrm{sub}(#1)}
\newcommand{\en}[1]{\mathsf{en}(#1)}
%\newcommand{\en}[1]{\mathrm{en}(#1)}
\newcommand{\cuenta}[2]{\mathsf{cuenta}\ensuremath{(#1, #2)}}
%\newcommand{\cuenta}[2]{\ensuremath{\mathrm{cuenta}(#1, #2)}}
\newcommand{\suma}[1]{\mathsf{suma}(#1)}
%\newcommand{\suma}[1]{\mathrm{suma}(#1)}
\newcommand{\twodots}{\mathrm{..}}

% Acumulador
\newcommand{\acum}[1]{\mathsf{acum}(#1)}

% \selector{variable}{dominio}
\newcommand{\selector}[2]{#1~\ensuremath{\leftarrow}~#2}
\newcommand{\selec}{\ensuremath{\leftarrow}}

% -----------------
% Especificacion
% -----------------

% Para problemas con resultado:
% \begin{problema}{nombre}{argumentos}{resultado}
%       \modifica{variables}
%       \requiere[nombre]{condicion}
%       \requiere[nombre]{condicion}
%       ...
%       \asegura[nombre]{condicion}
%       \asegura[nombre]{condicion}
%       ...
% \end{problema}

% Para problemas sin resultado:
% \begin{problema*}{nombre}{argumentos}
%       \modifica{variables}
%       \requiere[nombre]{condicion}
%       \requiere[nombre]{condicion}
%       ...
%       \asegura[nombre]{condicion}
%       \asegura[nombre]{condicion}
%       ...
% \end{problema*}

\newenvironment{problema}[3]{
    \vspace{0.2cm}
    \noindent \textsf{problema #1}\ensuremath{(#2) = #3\{}\\
    \begin{tabular}{p{0.02\textwidth} p{0.85 \textwidth}}
}{
    \end{tabular}

    \noindent \ensuremath{\}}
    \vspace{0.15cm}
}

\newenvironment{problema*}[2]{
    \vspace{0.2cm}
    \noindent \textsf{problema #1}\ensuremath{(#2)\{}\\
    \begin{tabular}{p{0.02\textwidth} p{0.85 \textwidth}}
}{
    \end{tabular}

    \noindent \ensuremath{\}}
    \vspace{0.15cm}
}

\newcommand{\requiere}[2][]{& \textsf{requiere #1: }\ensuremath{#2};\\}
\newcommand{\asegura}[2][]{& \textsf{asegura #1: }\ensuremath{#2};\\}
\newcommand{\modifica}[1]{& \textsf{modifica }\ensuremath{#1};\\}
\newcommand{\pre}[1]{\textsf{pre}\ensuremath{(#1)}}
\newcommand{\aux}[2]{& \textsf{aux }\ensuremath{#1 = #2};\\}
\newcommand{\problemanom}[1]{\textsf{#1}}
\newcommand{\problemail}[3]{\textsf{problema #1}\ensuremath{(#2) = #3}}
\newcommand{\problemailsinres}[2]{\textsf{problema #1}\ensuremath{(#2)}}
\newcommand{\requiereil}[2][]{\textsf{requiere #1: }\ensuremath{#2}}
\newcommand{\asegurail}[2][]{\textsf{asegura #1: }\ensuremath{#2}}
\newcommand{\modificail}[1]{\textsf{modifica }\ensuremath{#1}}
\newcommand{\auxil}[2]{\textsf{aux }\ensuremath{#1 = #2};}
\newcommand{\auxnom}[1]{\textsf{aux }\ensuremath{#1}}

% -----------------
% Tipos compuestos
% -----------------

\newcommand{\Pred}[1]{\mathit{#1}}
\newcommand{\TSet}[1]{\textsf{Conjunto}\ensuremath{\langle #1 \rangle}}
\newcommand{\TSetFinito}[1]{\textsf{Conjunto}\ensuremath{\langle #1 \rangle}}
\newcommand{\TRac}{\tiponom{Racional}}
\newcommand{\TVec}{\tiponom{Vector}}
\newcommand{\True}{\mathrm{True}}
\newcommand{\False}{\mathrm{False}}
\newcommand{\Func}[1]{\mathrm{#1}}
\newcommand{\cardinal}[1]{\left| #1 \right|}
\newcommand{\enum}[2]{\ensuremath{\mathsf{#1} = \langle \mathsf{#2} \rangle}}

\newenvironment{tipo}[1]{%
    \vspace{0.2cm}
    \textsf{tipo #1}\ensuremath{\{}\\
    \begin{tabular}[l]{p{0.02\textwidth} p{0.02\textwidth} p{0.82 \textwidth}}
}{%
    \end{tabular}

    \ensuremath{\}}
    \vspace{0.15cm}
}

\newcommand{\observador}[3]{%
    & \multicolumn{2}{p{0.85\textwidth}}{\textsf{observador #1}\ensuremath{(#2):#3}}\\%
}
\newcommand{\observadorconreq}[3]{
    & \multicolumn{2}{p{0.85\textwidth}}{\textsf{observador #1}\ensuremath{(#2):#3 \{}}\\
}
\newcommand{\observadorconreqfin}{
    & \multicolumn{2}{p{0.85\textwidth}}{\ensuremath{\}}}\\
}
\newcommand{\obsrequiere}[2][]{& & \textsf{requiere #1: }\ensuremath{#2};\\}

\newcommand{\explicacion}[1]{&& #1 \\}
\newcommand{\invariante}[1]{%
    & \multicolumn{2}{p{0.85\textwidth}}{\textsf{invariante }\ensuremath{#1}}\\%
}
\newcommand{\auxinvariante}[2]{
    & \multicolumn{2}{p{0.85\textwidth}}{\textsf{aux }\ensuremath{#1 = #2}};\\
}

\newcommand{\tiponom}[1]{\ensuremath{\mathsf{#1}}\xspace}
\newcommand{\obsnom}[1]{\ensuremath{\mathsf{#1}}}

% -----------------
% Ecuaciones de terminacion en funcional
% -----------------

\newenvironment{ecuaciones}{%
    $$
    \begin{array}{l @{\ /\ (} l @{,\ } l @{)\ =\ } l}
}{%
    \end{array}
    $$
}

\newcommand{\ecuacion}[4]{#1 & #2 & #3 & #4\\}

% Listas por comprension. El primer parametro es la expresion y el
% segundo tiene los selectores y las condiciones.
\newcommand{\comp}[2]{[\,#1\,|\,#2\,]}
