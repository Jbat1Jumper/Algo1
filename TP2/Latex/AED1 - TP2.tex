% Clase y configuracion de tipo de documento
\documentclass[10pt,a4paper,spanish]{article}
% Inclusion de paquetes
\usepackage{a4wide}
\usepackage{amsmath, amscd, amssymb, amsthm, latexsym}
\usepackage[spanish]{babel}
\usepackage[utf8]{inputenc}
\usepackage[width=15.5cm, left=3cm, top=2.5cm, height= 24.5cm]{geometry}
\usepackage{fancyhdr}
\pagestyle{fancyplain}
\usepackage{listings}
\usepackage{enumerate}
\usepackage{xspace}
\usepackage{longtable}
\usepackage{caratula}
% incluye macros espec materia
\include{algo1-cmds}

% Encabezado
\lhead{Algoritmos y Estructuras de Datos I}
\rhead{Grupo 4}
% Pie de pagina
\renewcommand{\footrulewidth}{0.4pt}
\lfoot{Facultad de Ciencias Exactas y Naturales}
\rfoot{Universidad de Buenos Aires}

\begin{document}

% Datos de caratula
\materia{Algoritmos y Estructuras de Datos I}
\titulo{Trabajo Pr\'actico N\'umero 2}
\subtitulo{Especificaci\'on, Demostraci\'on e Implementaci\'on}
\grupo{Grupo: 4}

\integrante{Aun Castells, Mar\'ia Virginia}{366/13}{vauncastells@hotmail.com}
\integrante{Motta, Leandro}{85/14}{leamotta@msn.com}
\integrante{Zdanovitch, Nikita}{520/14}{3hb.tch@gmail.com}
\integrante{de Monasterio, Francisco}{764/13}{franciscodemonasterio@outlook.com}

\maketitle

\newpage

% Para crear un indice
%\tableofcontents

% Forzar salto de pagina
\clearpage

% Pueden modularizar el documento incluyendo otros .tex
% \include{observaciones}

%\section{Observaciones}
%
%	\begin{enumerate}
%		\item un item
%		\item otro item
%	\end{enumerate}
%
% Otro salto de pagina
% \newpage

\section{Resolución}

\begin{ejercicio}
	% EJEMPLO
	Posiciones M\'as Oscuras:

	\begin{problema}{posicionesM\'asOscuras}{i: Imagen}{res:[(\ent,\ent)]}		

		\aseguraml[pertenecenALaImagen]{
			} \nuevalinea{( \forall\ pos \selec res)\ 0 \leq \prm{pos} < ancho(i)\  \land\ 0 \leq \sgd{pos} < alto(i);
		}
		\aseguraml[tienenMinimaLuminosidad]{} \nuevalinea{( \forall\ pos \selec res)\ lum(color(i,\prm{pos} ,\sgd{pos})) == minLum(i);}
		%\nuevalinea{ res[i][j] == medianaONegro(i,\ j,\ img,\ k); }
	\end{problema}
	
\end{ejercicio}

\begin{ejercicio}
	% EJEMPLO
	Top 10:

	\begin{problema}{top10}{g: Galeria}{res:[Imagen]}		

		\requiere[]{
			|imagenes(g)| \geq 10
			}
		\aseguraml[]{res == primeros(10,invertir(imagenesEnOrden(g)));}
	\end{problema}
	
\end{ejercicio}

\begin{ejercicio}
	% EJEMPLO
	La M\'as Chiquita Con Punto Blanco:

	\begin{problema}{laMasChiquitaConPuntoBlanco}{g: Galeria}{res:Imagen}		

		\requiere[]{
			|imagenes(g)| > 0
			}
		\aseguraml[]{esMenor(conPixelBlancos(g));}
		%\nuevalinea{ res[i][j] == medianaONegro(i,\ j,\ img,\ k); }
	\end{problema}
	
\end{ejercicio}

\begin{ejercicio}
	% EJEMPLO
	Agregar Imagen:

	\begin{problema}{agregarImagen}{g: Galeria, i: Imagen}{}
		\modifica{g}
		\asegura[]{
			|imagenes(\pre{g})|+ 1 == |imagenes(g)|
		}
		\aseguraml{
			mismos(imagenes(\pre{g}++i,\ imagenes(g));
		}		
%		\aseguraml{
%			\left( \forall\ y \selec {[0\,..\,alto(imagen))} \right) \left( \forall\ x \selec {[0\,..\,ancho(imagen))} \right) 
%		}
%		\nuevalinea{ \IF   esKCompleto(kVecinos(imagen,\ y,\ x,\ k),\ k)\ }
%		\nuevalinea{ \THEN esPromedio(res, imagen,\ y,\ x,\ k)\ }
%		\nuevalinea{ \ELSE esNegro(res,\ y,\ x) ;}
	\end{problema}

\end{ejercicio}

\begin{ejercicio}
	% EJEMPLO
	Votar:

	\begin{problema}{votar}{g: Galeria, i: Imagen}{}
		\modifica{g}
		\requiere[]{
			|imagenes(g)| > 0
			}
		\asegura[]{
			votos(g,i) == votos(\pre{g},i) + 1
		}
		\aseguraml{
			mismos(imagenes(\pre{g},\ imagenes(g));
		}		
		\aseguraml{
			( \forall\ j \selec imagenes(g), j\neq i)\ votos(g,j) == votos(\pre{g},j);
		}
%		\nuevalinea{ \IF   esKCompleto(kVecinos(imagen,\ y,\ x,\ k),\ k)\ }
%		\nuevalinea{ \THEN esPromedio(res, imagen,\ y,\ x,\ k)\ }
%		\nuevalinea{ \ELSE esNegro(res,\ y,\ x) ;}
	\end{problema}

\end{ejercicio}

\begin{ejercicio}
	% EJEMPLO
	eliminarMasVotada:

	\begin{problema}{eliminarMasVotada}{g: Galeria}{}
		\modifica{g}
		\requiere[]{
			|imagenes(g)| > 0
			}
		\asegura[]{
			|imagenes(\pre{g})|- 1 == |imagenes(g)|
		}
		\asegura[lasNuevasEstaban]{
		} \nuevalinea{( \forall\ x \selec imagenes(g))\ x\ \epsilon\ imagenes(\pre{g}) \wedge votos(g,x) == votos(\pre{g},x);}
		\asegura[peroFaltaUnaDeLasMasVotadas]{
			\neg mismos(masVotados(\pre{g},\ masVotados(g))
		}		
	\end{problema}

\end{ejercicio}


%\begin{ejercicio}
%	% EJEMPLO
%	Otro ejemplo sin resultado:
	
%	\begin{problema*}{cociente}{a,b:\ent}
%		\requiere{b \neq 0}
%		\modifica{a,b}
%		\asegura{a == pre(a)$ $div$ $pre(b)}
%		\asegura{b == pre(a)$ $mod$ $pre(b)}
%	\end{problema*}

%\end{ejercicio}

% \subsection{Ejercicio X}

\subsection{Auxiliares}

%\begin{itemize}
%	\item \auxil{Alto(img:{{(\ent ,\ent ,\ent)}}} : \ent}\longitud{img}}
%\end{itemize}

\begin{itemize}

	\item \auxilml{colores(i: Imagen) : [Pixel]} {} 
		\auxnl{
		[color(i,x,y)\ |\ x \selec [0..ancho(i)),\ y \selec [0..alto(i))];
		}

	\item \auxil{lum(p: Pixel) : \ent}{
		red(p) + blue(p) + green(p)
	}
	
	\item \auxil{minLum(i: Imagen) : \ent}{
		lum(\cab{minLumPixels(i)})
	}
	
	\item \auxilml{minLumPixels(i: Imagen) : [Pixel]} {} 
		\auxnl{
		[c\ |\ c \selec colores(i), (\forall c2 \selec colores(i))\ lum(c) \leq lum(c2)];
		}
		
	\item \auxil{primeros(n:\ent, l: [T]) : [T]}{
		\IF |l| \leq n \THEN l \ELSE [l[i]\ |\ i \selec [0..10)]
	}	
			
	\item \auxilml{invertir(l: [T]) : [T]} {} 
		\auxnl{
			[l[|l|-i]\ |\ i \selec (0..|l|]];
		}		
		
	\item \auxilml{imagenesEnOrden(g: Galeria) : [Imagen]} {} 
		\auxnl{
			[i\ |\ v \selec [0..maxVoto(g)],\ i \selec imagenes(g), votos(g,i) == v];
		}		
			
	\item \auxilml{maxVoto(g: Galeria) : \ent}{
		\IF |imagenes(g)| == 0 \THEN 0 \ELSE }
		\auxnl{
			\cab{[ votos(g,i)\ |\ i \selec imagenes(g), (\forall i2 \selec imagenes(g)) votos(g,i) \geq votos(g,i2)]};
		}		
	
	\item \auxilml{esMenor(imagenes: [Imagen],imagen: Imagen) : \bool} {
	} \auxnl{
		( \forall\ i \selec imagenes)\ ancho(i)\geq ancho(imagen) \wedge alto(i) \geq alto(imagen);
		}	
	
	\item \auxilml{conPixelBlanco(g: Galeria) : [Imagen]} {} 
		\auxnl{
			[imagen\ |\ imagen \selec imagenes(g),\ pixelBlanco(imagen) == true];
		}
		
		\item \auxilml{pixelBlanco(imagen: Imagen) : \bool} { 
		No\ Entendi\ esta\ y\ creo\ que\ no\ se\ puede\ hacer\ de\ esta\ manera
		} 
	
	\item \auxilml{masVotados(g: Galeria) : [Imagen]} {} 
		\auxnl{
		[img\ |\ img \selec imagenes(g),\ masVotado(g,img)];
		}

	\item \auxilml{masVotado(g: Galeria,img: Imagen) : \bool} {
	} \auxnl{
		( \forall\ imaggal \selec imagenes(g))\ votos(g,imaggal) \geq votos(g,img);
		}

	\item \auxilml{esImagenValida(img:[[(\ent,\ent,\ent)]]) : \bool}{ esRectangular(imagen)\ \land\  }
	\auxnl{ 
		(\forall\ f \selec imagen) (\forall\ px \selec f)\ esPixelValido(px);
	}
	
	\item \auxil{alto(img:[[(\ent,\ent,\ent)]]) : \ent}{
		|img|
	}
	\item \auxil{ancho(img:[[(\ent,\ent,\ent)]]) : \ent}{
		\IF |img| == 0 \THEN 0 \ELSE |\cab{img}|
	}
	
	\item \auxil{pixel(img:[[(\ent,\ent,\ent)]],\ y,x:\ent) : (\ent,\ent,\ent)}{
		 \IF esIndiceValido(y,\ x,\ img) \THEN img[y][x] \ELSE (0,\ 0,\ 0)
	}
	
	\item \auxil{cuenta(x:T,\ a:[T]) : \ent}{
		\big|[ y\ |\ y \selec a,\ y == x ]\big|
	}
	
	\item \auxil{mismos(a,b:[T]) : \bool}{
		(|a| == |b|) \land ( \forall\ c \selec a)\ cuenta(c,\ a) == cuenta(c,\ b)
	}

\end{itemize}

        
\end{document}
